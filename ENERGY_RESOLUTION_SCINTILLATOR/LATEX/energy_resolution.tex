\documentclass[a4paper]{article}
\usepackage[utf8]{inputenc}
\usepackage{geometry}
\usepackage{amsmath}
\pdfpagewidth
\paperwidth
\pdfpageheight
\paperheight
\usepackage{booktabs}
\usepackage{graphicx}
\usepackage{subfig}
\usepackage{verbatim}
\newcommand*{\unit}[1]{\ensuremath{\mathrm{\,#1}}}
\usepackage{amsthm}
\usepackage{epsfig}
\usepackage{fancyhdr} 
\usepackage{amsmath,amssymb}
\usepackage{amscd} 
\usepackage[T1]{fontenc} 
\usepackage[utf8]{inputenc} 
\usepackage[usenames,dvipsnames]{xcolor}
\usepackage{graphicx,color,listings}
\usepackage{hologo}
\frenchspacing 
\usepackage{float}
\usepackage{geometry}
\usepackage{rotating}
\usepackage{caption}
\captionsetup{labelformat=empty, textfont=sl}
\usepackage{placeins}
\usepackage{hyperref}
\usepackage{listings}
\frenchspacing
\title{Esperienza Laboratorio di Fisica Medica: Esercizio di stima della risoluzione energetica di rivelatori a scintillazione}
\author{Jake Harold Pensavalle, Lorenzo Marini, Simone Lossano}
\begin{document}
	\maketitle
	\newpage
	\tableofcontents
	\newpage
%%%%%%%%%%%%%%%%%%%%%%%%%%%%%%%%%%%%%%
\section{Abstract}
%%%%%%%%%%%%%%%%%%%%%%%%%%%%%%%%%%%%%%


%%%%%%%%%%%%%%%%%%%%%%%%%%%%%%%%%%%%%%
\section{Introduzione}
%%%%%%%%%%%%%%%%%%%%%%%%%%%%%%%%%%%%%%


%%%%%%%%%%%%%%%%%%%%%%%%%%%%%%%%%%%%%%
\section{Analisi Dati}
%%%%%%%%%%%%%%%%%%%%%%%%%%%%%%%%%%%%%%


%%%%%%%%%%%%%%%%%%%%%%%%%%%%%%%%%%%%%%
\subsection{Analisi con Python}
%%%%%%%%%%%%%%%%%%%%%%%%%%%%%%%%%%%%%%


%%%%%%%%%%%%%%%%%%%%%%%%%%%%%%%%%%%%%%
\section{Risultati}
%%%%%%%%%%%%%%%%%%%%%%%%%%%%%%%%%%%%%%



%%%%%%%%%%%%%%%%%%%%%%%%%%%%%%%%%%%%%%
\subsection{Gaussiana e Gaussiana con fondo lineare}
%%%%%%%%%%%%%%%%%%%%%%%%%%%%%%%%%%%%%%


%%%%%%%%%%%%%%%%%%%%%%%%%%%%%%%%%%%%%%
\subsection{Gaussiana con fondo lineare e code esponenziali}
%%%%%%%%%%%%%%%%%%%%%%%%%%%%%%%%%%%%%%


%%%%%%%%%%%%%%%%%%%%%%%%%%%%%%%%%%%%%%
\subsection{Gaussiana con Klein-Nishina e fondo lineare}
%%%%%%%%%%%%%%%%%%%%%%%%%%%%%%%%%%%%%%


%%%%%%%%%%%%%%%%%%%%%%%%%%%%%%%%%%%%%%
\section{Confronto con la risoluzione reale}
%%%%%%%%%%%%%%%%%%%%%%%%%%%%%%%%%%%%%%


\end{document}